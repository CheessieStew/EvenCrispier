\chapter{Wstęp}
W ramach tej pracy stworzyłem grę przedstawiającą uproszczony model środowiska naturalnego, której przeznaczeniem jest testowanie metod sztucznej inteligencji. Aplikacja pozwala kontrolować stan symulacji i obserwować, jak AI uczy się grać optymalnie.

Agentem w tej grze jest osiołek, którego zadaniem jest przeżycie odżywiając się ostem.

\section{Motywacja}
Założeniem projektu było stworzenie symulacji inspirowanej światem rzeczywistym, w której egzystują byty zdolne do nauczenia się strategii pozwalającej na optymalne funkcjonowanie. 


\section{Realizacja}
Na potrzeby realizacji celu powstały trzy elementy:

\begin{enumerate}
    \item Środowisko graficzne służące do wizualnego reprezentowania i kontrolowania symulacji
    \item Silnik symulacji, będący grą dla jednego gracza, w której osiołek (gracz) ma za zadanie przeżyć jak najdłużej odżywiając się roślinami
    \item Ucząca się sztuczna inteligencja kontrolująca osiołka
\end{enumerate}

\section{Narzędzia}
Projekt w całości został zrealizowany w języku C\# z wykorzystaniem silnika WPF do części wizualnej.


