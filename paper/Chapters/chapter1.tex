\chapter{Wstęp}
W ramach tej pracy stworzyłem grę przedstawiającą uproszczony model środowiska naturalnego, której przeznaczeniem jest testowanie metod sztucznej inteligencji. Aplikacja pozwala kontrolować stan symulacji i obserwować, jak AI uczy się grać optymalnie.

Agentem w tej grze jest osiołek, którego zadaniem jest przeżycie odżywiając się ostem. Poczynaniami osiołka kieruje kontroler oparty o algorytm Q-learning \cite{wiki:Qlearning}, który jest odmianą uczenia ze wzmocnieniem (ang. \textit{reinforcement learning}) \cite{Sutton1998Introduction}. Metoda ta polega na uczeniu się optymalnych zachowań przez analizę nagród i kar płynących z podejmowanych decyzji i jest ważną gałęzią uczenia maszynowego \cite{Sutton1988Learning}.  

Wartymi wspomnienia przykładami zastosowania uczenia ze wzmocnieniem są na przykład TD-Gammon \cite{Tesauro1994TDGammon}, który nauczył się grać w grę Backgammon w oparciu o jedynie wyniki gier rozgrywanych z samym sobą, czy AlphaGo, który w 2016 roku był najlepszym programem grającym w Go \cite{Silver2016Mastering}, a jego ostatnia wersja korzystająca z czystego RL (bez wiedzy ludzkich ekspertów wykorzystywanej przez wersję z 2016) wygrywa z poprzednią 100 na 100 gier \cite{silver2017mastering}. Innym znanym zastosowaniem uczenia ze wzmocnieniem jest gra w szachy -- przykładem jest program KnightCap wykorzystujący algorytm TDLeaf($\lambda$) \cite{Baxter2000Learning}.


\section{Motywacja}
Założeniem projektu było stworzenie symulacji inspirowanej światem rzeczywistym, w której egzystują byty zdolne do nauczenia się strategii pozwalającej na optymalne funkcjonowanie. Stworzony model powinien nie tylko pozwolić sztucznej inteligencji na naukę, ale też, po nauczeniu, pozwalać na zaobserwowanie ciekawych zachowań.


\section{Realizacja}
Na potrzeby realizacji celu powstały trzy elementy:

\begin{enumerate}
    \item Środowisko graficzne służące do wizualnego reprezentowania i kontrolowania symulacji
    \item Silnik symulacji, będący grą dla jednego gracza, w której osiołek (gracz) ma za zadanie przeżyć jak najdłużej odżywiając się roślinami
    \item Ucząca się sztuczna inteligencja kontrolująca osiołka
\end{enumerate}

\section{Narzędzia}
Projekt w całości został zrealizowany w języku C\# z wykorzystaniem silnika WPF do części wizualnej.


