\chapter{Zasady symulacji}
Symulacja stworzona na potrzeby tej pracy to prosta, turowa gra dla jednego gracza rozgrywająca się na ciągłej, dwuwymiarowej, kwadratowej planszy, na której żyje osiołek. Osiołek musi odżywiać się ostem by przeżyć, lecz poruszanie się po planszy go męczy, powodując dodatkowe zapotrzebowanie na pożywienie.

\section{Byty}
Wszystkie byty posiadają następujące cechy:
\begin{enumerate}    
    \item Pozycja na planszy
    \item Identyfikator
\end{enumerate}

\subsection{Osiołek}
Osiołek jest bytem kontrolowanym przez gracza. W danej turze na planszy znajduje się dokładnie jeden osiołek. Osiołek posiada następujące cechy zdefiniowane w ustawieniach [sekcja \ref{fig:example_settings}]:
\begin{enumerate}    
    \item \textbf{Mass} -- masa, która opisuje stopień wypełnienia żołądka i umożliwiająca osiołkowi funkcjonowanie poprzez spalanie jej
    \item \textbf{MovementSpeed} -- maksymalny dystans, który może pokonać w ciągu tury
    \item \textbf{InteractionDistance} -- zasięg, na którym możliwe jest jedzenie ostu
    \item \textbf{BiteSize} -- maksymalna masa, którą może przyswoić z rośliny w ciągu tury
    \item \textbf{StomachCapacity} -- maksymalny rozmiar żołądka
    \item \textbf{SightRange} -- zasięg, w którym inne byty są dla osiołka widoczne
    \item \textbf{PassiveWork} -- ilość masy zużywanej co turę
    \item \textbf{MovementWork} -- ilość masy zużywanej na potrzeby ruchu 
\end{enumerate}

\subsection{Oset}
Osiołek żywi się ostem. Na planszy może znajdować się co najwyżej określona w ustawieniach symulacji liczba roślin. Każdy oset ma nastepujące cechy:
\begin{enumerate}    
    \item \textbf{Mass} -- określa ilość dostępnego pożywienia w danej roślinie
    \item \textbf{MaxMass} -- maksymalna masa, do której roślina może urosnąć
    \item \textbf{RegrowthRate} -- stały przyrost masy na turę
\end{enumerate}
Maksymalna masa i przyrost masy na turę danej rośliny są losowane z rozkładem jednostajnym na przedziałach zdefiniowanych w ustawieniach [sekcja \ref{fig:example_settings}]. Początkowa masa ostu równa jest jego masie maksymalnej.

\section{Początek gry}
Przed pierwszą turą w losowych miejscach tworzone jest tyle roślin, ile wskazują ustawienia. Następnie w losowym miejscu umieszczany jest osiołek, którego początkowa masa wynosi połowę maksymalnej.

\section{Przebieg gry}
Każda tura przebiega następująco:
\begin{enumerate}    
    \item Wszystkie rośliny, których obecna masa jest większa niż zero zwiększają swoją masę w oparciu o \textbf{RegrowthRate}, nie przekraczając \textbf{MaxMass}.
    \item Wszyskie byty, których masa nie przekracza zera umierają i znikają. Jeśli umiera osiołek symulacja jest resetowana.
    \item Osiołek otrzymuje opis otoczenia i swojego stanu.
    \item Kontroler wybiera akcję dla osiołka w oparciu o ten opis.
    \item Osiołek wykonuje akcję i spala część swojej masy określoną wzorem:

     $\textbf{PassiveWork} + \textbf{MovementWork} \cdot \frac{\textit{distance}}{\textbf{MovementSpeed}}$, gdzie \textit{distance} to dystans przebyty w danej turze.
    \item Jeżeli liczba roślin na planszy nie przekracza maksymalnej, w losowym miejscu na planszy pojawia się nowa roślina, nie częściej niż raz na liczbę tur opisaną w ustawieniach.
\end{enumerate}

\section{Akcje}
W każdej turze siołek może wykonać jedną z poniższych akcji:
\begin{enumerate}    
    \item Pominięcie tury.
    \item Próba zjedzenia wybranej rośliny, która przekazuje część swojej masy osiołkowi jeśli ten znajduje się w odpowiednim zasięgu. Ilość zaabsorbowanej masy wynosi.
    
    $min(\textbf{BiteSize}, \textbf{plant.Mass}, \textbf{StomachCapacity - donkey.Mass})$
    \item Poruszenie się w stronę zadanego punktu o wybrany dystans z zakresu $[0,\textbf{MovementSpeed}]$.
\end{enumerate}



