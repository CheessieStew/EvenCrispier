\polishabstract{Metody sztucznej inteligencji znajdują zastosowanie w wielu dziedzinach, jak analizowanie zachowań użytkowników stron internetowych, kontrolowanie autonomicznych samochodów czy granie w gry (np. szachy). W tej pracy pokazuję wykorzystanie algorytmu Q-Learning do stworzenia symulacji modelującej w uproszczony sposób zachowanie roślinożercy w środowiskach pozbawionych drapieżników. W tym celu zaimplementowałem wizualizację i model środowiska oraz przetestowałem różne parametry algorytmu względem wielu ustawień świata, w którym uczy się on żyć. }

\englishabstract{Artificial intelligence has many applications, such as analyzing behaviours of website users, controlling self-driving cars or playing games (such as chess). In this thesis, I present a way of using the Q-Learning algorithm to create a simulation modelling the behaviour of an herbivore in a predator--free environment. I created a visualisation and the environment's model, and tested various parameters of the algorithm in the context of many example world settings.}