\polishabstract{Metody sztucznej inteligencji znajdują zastosowanie w wielu dziedzinach, jak analizowanie zachowań użytkowników stron internetowych, kontrolowanie autonomicznych samochodów czy granie w gry (np. szachy). W tej pracy pokazuję wykorzystanie algorytmu Q-Learning do stworzenia symulacji modelującej w uproszczony sposób zachowanie roślinożercy w środowiskach pozbawionych drapieżników. W tym celu zaimplementowałem wizualizację i model środowiska oraz przetestowałem różne parametry algorytmu względem wielu ustawień świata, w którym uczy się on żyć. }

